%% start of file `template_en.tex'.
%% Copyright 2006-1008 Xavier Danaux (xdanaux@gmail.com).
%
% This work may be distributed and/or modified under the
% conditions of the LaTeX Project Public License version 1.3c,
% available at http://www.latex-project.org/lppl/.


\documentclass[11pt,a4paper]{moderncv}

% moderncv themes
\moderncvtheme[blue]{casual}                 % optional argument are 'blue' (default), 'orange', 'red', 'green', 'grey' and 'roman' (for roman fonts, instead of sans serif fonts)
%\moderncvtheme[green]{classic}                % idem

% character encoding
\usepackage[utf8]{inputenc}                   % replace by the encoding you are using

\usepackage[english,ngerman]{babel}          % deutschsprachige Bezuege
% \usepackage[english]{babel}

% adjust the page margins
\usepackage[scale=0.8]{geometry}
%\setlength{\hintscolumnwidth}{3cm}						% if you want to change the width of the column with the dates
\usepackage{setspace}           %for paragraphing cv sections:
                                %wrap section lines in {\setstretch{0.6} ... }
                                %insert empty cvline/cventry/... between
                                %paragraphs

\usepackage{makecell}           %for manual linebreaks within table cells,
                                % reference:
                                % https://tex.stackexchange.com/a/176780

\usepackage{array}
\usepackage{csquotes}           % language-aware quotes: \enquote{}

% redefine moderncv section bar color, called 'color1'.
% reference: https://stackoverflow.com/a/64538448/8116031
\usepackage{xcolor}
\definecolor{nomadblue}{HTML}{00B7FD}
\colorlet{color1}{nomadblue}

%\AtBeginDocument{\setlength{\maketitlenamewidth}{6cm}}  % only for the classic theme, if you want to change the width of your name placeholder (to leave more space for your address details
\AtBeginDocument{\recomputelengths}                     % required when changes are made to page layout lengths

% personal data
\firstname{Johannes}
\familyname{Wasmer}
%\title{Resumé title (optional)}               % optional, remove the line if not wanted
\address{Rütscher Str. 155 Z.0102}{DE-52072 Aachen}    % optional, remove the line if not wanted
\mobile{+49 151 5650 8795}                    % optional, remove the line if not wanted
% \phone{+49 241 8098 019}                      % optional, remove the line if not wanted
%\fax{07754 91086}                          % optional, remove the line if not wanted
\email{johannes.wasmer@gmail.com}
%\extrainfo{additional information (optional)} % optional, remove the line if not wanted
\photo[64pt]{picture_64pt.jpg}                         % '64pt' is the height the picture must be resized to and 'picture' is the name of the picture file; optional, remove the line if not wanted
%\quote{Some quote (optional)}                 % optional, remove the line if not wanted
\social[linkedin][linkedin.com/in/johannes-wasmer-359749126/]{johannes-wasmer}
\social[github][github.com/Irratzo/about-myself/blob/master/README.md]{Irratzo}
\social[twitter]{JohannesWasmer}

%\nopagenumbers{}                             % uncomment to suppress
%automatic page numbering for CVs longer than one page

%fuer C++ Name
\def\SymboCpp{C\raisebox{0.5ex}{\tiny\textbf{++}}}
% \def\CC{{C\nolinebreak[4]\hspace{-.05em}\raisebox{.4ex}{\tiny\bf ++}}}

%use inside document if needed:
%\renewcommand{\listitemsymbol}{-} % change the symbol for lists

\makeatletter
\renewcommand*{\makeletterclosing}{
  \@closing\\[3em]%
  \includegraphics{signatur2.png}\\% Insert signature
  {\bfseries \@firstname~\@lastname}%
  \ifthenelse{\isundefined{\@enclosure}}{}{%
    \\%
    \vfill%
    {\color{color2}\itshape\enclname: \@enclosure}}}
\makeatother


%----------------------------------------------------------------------------------
%            content
%----------------------------------------------------------------------------------
\begin{document}
\sffamily   % for use with a résumé using sans serif fonts;
\maketitle

%cventry: % arguments 3 to 6 are optional

\section{Bildungsweg}

\cventry{2017/10 --
  2022/03}{\href{https://www.aices.rwth-aachen.de/en/academics/masters-program-simulation-sciences}{Simulation
    Sciences MSc}}{\href{https://www.rwth-aachen.de}{RWTH}}{Aachen}{\textit{Note 2,0 (Stand Jan. 2022)}}{Individueller Schwerpunkt auf
  datengetriebener Materialwissenschaft. Abschlussarbeit: \enquote{Entwicklung eines
    Ersatzmodells aus dem Maschinellen Lernen zur Beschleunigung von
    Dichtefunktional-Rechnungen mit der Korringa-Kohn-Rostoker-Methode}, betreut
  von \href{https://www.fz-juelich.de/SharedDocs/Personen/PGI/PGI-1/EN/Bluegel_S.html}{Prof.
    Dr. Stefan Blügel, PGI-1/IAS-1}, Forschungszentrum Jülich. }

\cventry{2014/09 -- 2017/10}
{\href{https://www.fh-aachen.de/en/course-of-study/angewandte-mathematik-und-informatik-bsc/}{Scientific
    Programming BSc}}{\href{https://www.fh-aachen.de/}{FH Aachen}}{Jülich/Aachen}{\textit{Note 2,1}}
{Duales Studium, kombiniert mit dem Ausbildungsberuf
  \enquote{\href{http://www.itc.rwth-aachen.de/cms/IT-Center/Lehre-Ausbildung/~letj/MATSE-Ausbildung/}{Mathematisch-Technischer
      Softwareentwickler (MATSE)}}. Ausbildung beim unabhängigen
  Forschungszentrum \href{https://access-technology.de/}{Access
    e.V., Aachen}. Abschlussarbeit: \enquote{Umstellung auf JavaFX und
    Erweiterung um ein Tool zur statistischen Datenanalyse einer
    Simulationsverwaltungs-Software} (MySQL, Hibernate, Java 8).}

\cventry{2004--2014}{Verschiedene}{\href{https://www.tum.de/}{TU München},
  \href{https://www.ethz.ch/de.html}{ETH Zürich}}{}{\textit{ohne Abschluss}}
{Studien in Natur-, Material-, und Rechnergestützen Wissenschaften. Kein Abschluss aufgrund einer 2013/2014 diagnostizierten und behandelten Depression.}

\cventry{1995--2004}{Abitur}{Hochrhein-Gymnasium}{Waldshut-Tiengen}{\textit{Note 2,1}}{Allgemeine Hochschulreife.}

\section{Berufserfahrung}

% \cventry{duration}{occupation}{company}{location}{empty}{description}

\cventry{2018/08-2018/10}{Praktikum}{Jülich Supercomputing
  Centre}{\href{https://www.fz-juelich.de}{Forschungszentrum
    Jülich}}{}{\href{https://www.fz-juelich.de/ias/jsc/EN/Career/Gueststudentprogramme/gsp_node.html}{Gaststudierendenprogramm
    2018}. Ich implementierte ein neues Visualisierungs-Plugin für das Performanzanalyse-Tool
  \href{https://www.vi-hps.org/tools/cube.html}{Cube}
  für parallele Programme (\SymboCpp{}14, Qt 5).}

\cventry{2017/10 -- 2018/09}{Studentische
  Hilfskraft}{\href{https://access-technology.de}{Access
    e.V.}}{Aachen}{}{Erstellung von Datenkonvertern als Bestandteil eines
  HDF5-Datenaustauschformats für das Projekt \enquote{\href{aixvipmap.de}{Aachen
      Virtual Platform for Materials Processing}} (Python).}{}

\cventry{2012/03 -- 2014/08}{Freier
  Journalist}{\href{http://www.suedkurier-medienhaus.de/}{Südkurier GmbH
    Medienhaus}}{Bad Säckingen}{{\small
    \href{https://www.suedkurier.de/services/info/Kontakt-Bad-Saeckingen;art1015385,4792765}{Redaktion
      Bad Säckingen}}}{}

% \cventry{2003 -- 2012}{Produktionshelfer}{\href{http://www.franke-personalservice.de/}{Franke Personalservice
% e.K.}}{Laufenburg}{}{{\small Ferienjobs.}}{}

\section{Sprachen}

\cvlanguage{Deutsch}{\href{http://www.europaeischer-referenzrahmen.de/}{C2 CEFR}}{\href{http://www.europaeischer-referenzrahmen.de/}{Muttersprache}}
\cvlanguage{Englisch}{\href{http://www.europaeischer-referenzrahmen.de/}{C1
    CEFR}}{\href{http://www.europaeischer-referenzrahmen.de/}{fachkundige Sprachkenntnisse (TOEFL iBT 2017: 110 von 120 Punkten)}}
\cvlanguage{Französisch}{\href{http://www.europaeischer-referenzrahmen.de/}{A2 CEFR}}{\href{http://www.europaeischer-referenzrahmen.de/}{elementare Sprachverwendung}}


\pagebreak

\section{IT-Kenntnisse}
% {\setstretch{0.6}
%   \cvcomputer{\scriptsize{OS}}{Linux, Windows}{\scriptsize{Office}}{MS Office, \LaTeX2e, LibreOffice}
%   \cvcomputer{}{}{}{}
%   \cvcomputer{{\scriptsize Programming}}{Python, \SymboCpp{}, Java 8 -- {\scriptsize\emph{advanced}}}{\scriptsize{APIs/}}{{\normalsize}
%     JavaFX/Qt, MPI/OpenMP/CUDA,}
%   \cvcomputer{{\scriptsize languages}}{Javascript -- {\scriptsize\emph{beginner}}}{{\scriptsize libraries}}{np/pd/tf, Hibernate/Spring}
%   \cvcomputer{}{}{}{}
%   \cvcomputer{{\scriptsize data handling}}{SQL, XML -- {\scriptsize\emph{intermediary}}}{{\scriptsize Web}}{HTML, CSS  -- {\scriptsize\emph{beginner}}}
%   \cvcomputer{}{RDF, MR/Spark -- {\scriptsize\emph{beginner}}}{}{}
%   \cvcomputer{}{}{}{}
%   \cvcomputer{{\scriptsize Dev tools/}}{Emacs/IDEs/Jupyter, VCS,}{{\scriptsize Multimedia}}{Gimp/Inkscape, Audacity (Audio),}
%   \cvcomputer{{\scriptsize methods}}{CMake/Gradle/Maven}{}{Kdenlive (Video) -- {\scriptsize\emph{beginner}}}
% }

\setlength{\tabcolsep}{5pt}
\begin{center}
  \resizebox{\textwidth}{!}{%
    % % simple table
    % \begin{tabular}{lllll}
    % % table with word-wrap
    % \begin{tabular}{p{0.13\linewidth} p{0.11\linewidth} p{0.22\linewidth} p{0.2\linewidth} p{0.25\linewidth}} % 0.23 + 0.11 + 0.22 + 0.2 + 0.25 = 0.33 + 0.67 = 1.0
    % % table with word-wrap, and prevent word breaks in specified columns. Allow individual cell linebreaks with \makecell.
    \begin{tabular}{p{0.13\linewidth} p{0.11\linewidth} >{\raggedright\arraybackslash}p{0.22\linewidth} >{\raggedright\arraybackslash}p{0.2\linewidth} >{\raggedright\arraybackslash}p{0.25\linewidth}} % 0.23 + 0.11 + 0.22 + 0.2 + 0.25 = 0.33 + 0.67 = 1.0
      Zeitachse & Sprachen & Bibliotheken & Tools & Anwendung\\
      \hline
      Fortlaufend & (Lisp) & Keine & \href{https://www.spacemacs.org/}{Emacs}, \href{https://orgmode.org/}{Org-mode} & Notizen\\
      2020-2021 Thesis & Python & \href{https://aiida.net}{AiiDA}, \href{https://judft.de}{JuKKR} (DFT code), \href{https://wiki.fysik.dtu.dk/ase/}{ASE}, \href{https://pymatgen.org/}{Pymatgen}, \href{https://docs.bokeh.org}{Bokeh}, \href{https://scikit-learn.org/}{Scikit-learn}, \href{https://singroup.github.io/dscribe/}{DScribe} & GitHub/GitLab CI/CD, \href{https://readthedocs.org/}{RTD}, \href{https://python-poetry.org}{Poetry}, \href{https://docs.pytest.org}{Pytest}, \href{https://slurm.schedmd.com/}{Slurm}, \href{https://www.gnu.org/software/screen/}{Screen} & Maschinelles Lernen -- Potenziale für DFT\\
      2017-2020 Studium & Python, C & \href{https://www.tensorflow.org/}{Tensorflow}, \href{https://keras.io}{Keras}, \href{https://www.mpi-forum.org/}{MPI}, \href{https://www.openmp.org/}{OpenMP}, \href{https://developer.nvidia.com/cuda-zone}{CUDA} & \href{https://colab.research.google.com/}{Google Colaboratory}, \href{https://www.comet.ml/}{Comet.ml}, \href{https://www.mathworks.com}{MATLAB}, \href{https://vtk.org/}{Vtk} & Neuronale Netzwerke, numerische Löser\\
      \makecell[l]{2019 \\ Freizeit} & Shell, JS & \href{https://www.meteor.com/}{Meteor} & \href{https://www.mongodb.com/}{MongoDB}, \href{https://pandoc.org}{Pandoc}, Wiki, (\href{https://www.docker.com/}{Docker}) & Event-Organizer, Studierenden-Wiki\\
      \makecell[l]{2018 \\ Praktikum} & \SymboCpp{}17 & Template-Programmierung / OO, \href{https://www.qt.io/}{Qt5}, \href{https://github.com/google/googletest}{Googletest} & \href{https://cmake.org/}{CMake}, Git, \href{https://texample.net/tikz/}{TiKZ}, \href{https://www.jetbrains.com/clion/}{CLion} & Treemapping, \href{https://www.scalasca.org/software/cube-4.x/}{Cube} (Performance-Explorer)\\
      \makecell[l]{2018 \\Studien- \\ projekt} & Python & Introspection / OO, \href{https://github.com/jupyter-widgets/ipywidgets}{Ipywidgets} (dashboarding), \href{https://www.h5py.org/}{h5py}, \href{https://matplotlib.org/}{Matplotlib}, \href{https://docs.python.org/3/library/tk.html}{Tkinter} & \href{https://jupyter.org/}{Jupyter} & Bandstruktur-Visualisierer\\
      \makecell[l]{2017-2018 \\ Arbeit} & Python & \href{https://www.hdfgroup.org/solutions/hdf5/}{HDF5} (Format), \href{https://numpy.org}{Numpy}, \href{https://pandas.pydata.org/}{Pandas}, \href{https://www.w3.org/OWL/}{RDF/OWL} (ontology) & \href{https://anaconda.org/conda-forge/}{Conda}, \href{https://protege.stanford.edu/}{Protégé}, \href{https://www.jetbrains.com/pycharm/}{PyCharm} & \href{https://micress.rwth-aachen.de/}{MICRESS} (Mikrostruktur-Simulation), \href{aixvipmap.de}{AiXViPMaP} (ICME-Plattform)\\
      2014-2017 Ausbildung & Java 8 & DI / MVVM (Design Patterns), \href{https://openjfx.io/}{JavaFX} (GUI), \href{https://hibernate.org/}{Hibernate} (ORM), \href{https://www.w3.org/TR/xml/}{XML}, UML, Batch script & \href{https://maven.apache.org/}{Maven}, \href{https://gradle.org/}{Gradle}, \href{https://www.apachefriends.org/index.html}{Xampp}, \href{https://subversion.apache.org/}{SVN}, \href{https://trello.com/}{Trello}, \href{https://www.putty.org/}{Putty}, \href{https://cygwin.com/}{Cygwin}, \href{https://www.eclipse.org/}{Eclipse}, \href{https://netbeans.apache.org//}{Netbeans}, Windows, \LaTeX{}, Office & DB-Manager, \href{https://www.plm.automation.siemens.com/global/en/products/simcenter/STAR-CCM.html}{Star-CCM+} (Fluiddynamik)\\
      2014-2017 Studium & \SymboCpp{}, JS & JQuery, LAMMPS & \LaTeX{}, VMD & Kursprojekte, MD-Simulationen\\
      Vor 2014 & Shell, \SymboCpp{} & Keine & \href{https://www.gromacs.org/}{GROMACS}, \href{https://www.audacityteam.org/}{Audacity}, \href{https://www.gimp.org}{Gimp}, \href{https://kdenlive.org}{Kdenlive} & Kunst\\
    \end{tabular}
  }
\end{center}



% \begin{tabular}{p{0.35\linewidth} | p{0.6\linewidth}}
%   Column 1  & Column2 \\ \hline
%   This text will be wrapped & Some more text \\
%   Some text here & This text maybe wrapped here if its tooooo long \\
% \end{tabular}
% \caption{Caption}
% \label{tab:my_label}


% \pagebreak

%\section{Master thesis}
%\cvline{title}{\emph{Title}}
%\cvline{supervisors}{Supervisors}
%\cvline{description}{\small Short thesis abstract}


%\section{Pflege- und Assistenzerfahrung}
%\cvline{Pflege}{In den letzten 5 Jahren unterstützte ich meine Eltern bei der Pflege meines Großvaters und meiner Großtante zuhause. In dieser Zeit waren beide auf einen Rollstuhl angewiesen.}

%\newpage

\section{Freizeit}

{\setstretch{0.6}
  \cvline{Gremien-}{2018-2020 Wohnheimsprecher des
    Studierenden-Wohnheim-Vereins
    \href{https://www.oph.rwth-aachen.de}{Otto-Petersen-}}
  \cvline{tätigkeit}{\href{https://www.oph.rwth-aachen.de}{Haus e.V.} mit 200
    Mitgliedern. Die Selbstverwaltung besitzt die technische
    Infrastruktur, unterhält Arbeitsgruppen, und unterstützt
    die Organisation zweier traditionsreicher Events, die zu den größten
    studentisch organisierten Events in Aachen zählen. Mein wichtigster Beitrag in
    diesem Amt war die Umwandlung der Selbstverwaltung in einen eingetragenen, gemeinnützigen Verein.}
  \cvline{Öffentlich-}{2015
    Aktivmitglied der Aachener Studenteninitiative
    \href{https://www.energybirds.org/}{Energybirds e.V.}
    zur} \cvline{keitsarbeit}{Förderung des öffentliches Verständnisses zur nachhaltigen
    Energietechnik und -politik.}
  \cvline{}{} \cvline{Musik}{1999-2015 Dirigent, Ausbilder und Aktivmitglied in
    verschiedenen Musikvereinen, Orchestern und Ensembles in Europa,
    hauptsächlich jedoch in Deutschland.}
}

%\cvline{hobby 2}{\small Description}
%\cvline{hobby 3}{\small Description}

\renewcommand{\listitemsymbol}{-} % change the symbol for lists

% \section{Sonstige Qualifikation}
% \cvlistitem{Gabelstapler-Führerschein}
%\cvlistitem{}
%\cvlistitem{}            % optional other symbol

%\section{Extra 2}
%\cvlistdoubleitem[\Neutral]{Item 1}{Item 4}
%\cvlistdoubleitem[\Neutral]{Item 2}{Item 5}
%\cvlistdoubleitem[\Neutral]{Item 3}{}

% Publications from a BibTeX file
%\nocite{*}
%\bibliographystyle{plain}
%\bibliography{publications}       % 'publications' is the name of a BibTeX file

% \makeletterclosing % insert signature

\end{document}


%% end of file `template_en.tex'.

%%% Local Variables:
%%% mode: latex
%%% TeX-master: t
%%% End:
